\documentclass[conference]{IEEEtran}
\IEEEoverridecommandlockouts
% The preceding line is only needed to identify funding in the first footnote. If that is unneeded, please comment it out.
\usepackage{cite}
\usepackage{amsmath,amssymb,amsfonts}
\usepackage{algorithmic}
\usepackage{graphicx}
\usepackage{textcomp}
\usepackage{xcolor}
\def\BibTeX{{\rm B\kern-.05em{\sc i\kern-.025em b}\kern-.08em
    T\kern-.1667em\lower.7ex\hbox{E}\kern-.125emX}}
\begin{document}

\title{Office Apocalypse Algorithm: Multi-Source Municipal Data Integration for NYC Office Building Vacancy Risk Prediction}

\author{\IEEEauthorblockN{Data Science Capstone Team}
\IEEEauthorblockA{\textit{Master's Data Science Program} \\
\textit{Pace University}\\
New York, NY, USA \\
email@pace.edu}
}

\maketitle

\begin{abstract}
Office building vacancy in New York City poses significant economic threats to urban tax revenue, neighborhood vitality, and commercial ecosystem stability. This paper presents the Office Apocalypse Algorithm, a novel machine learning framework that integrates six heterogeneous municipal datasets to predict office building vacancy risk at the building level. Our approach fuses property characteristics (PLUTO), transaction histories (ACRIS), transportation accessibility (MTA ridership), construction activity (DOB permits), business registry data, and storefront vacancy indicators to generate building-level risk scores indexed by Borough-Block-Lot (BBL) identifiers. The methodology combines feature engineering from temporal transaction patterns, spatial proximity analysis, and economic vitality indicators to create a comprehensive predictive model. Initial exploratory data analysis across 857,736 NYC buildings reveals strong correlations between transportation accessibility, transaction frequency, and vacancy risk patterns. The proposed gradient-boosted ensemble model with SHAP explainability provides stakeholders with actionable building-level risk assessments and interpretable feature contributions for policy intervention prioritization.
\end{abstract}

\begin{IEEEkeywords}
urban analytics, vacancy prediction, machine learning, municipal data integration, real estate risk modeling, explainable AI
\end{IEEEkeywords}

\section{Introduction}

Office building vacancy represents a critical urban economic challenge with cascading effects on municipal tax revenue, neighborhood commercial vitality, and regional economic stability. In New York City, the post-pandemic shift toward remote work has intensified concerns about office space utilization, creating urgent demand for predictive analytics to support proactive policy interventions and investment decisions.

The scientific question addressed in this research is: \textit{Can integration of heterogeneous municipal datasets enable accurate building-level prediction of office vacancy risk in NYC?} Traditional approaches to vacancy prediction rely on limited data sources, typically focusing on either building characteristics or economic indicators in isolation. This research advances the field by demonstrating meaningful integration of six distinct municipal datasets to create comprehensive risk assessments.

The primary objectives of this study are: (1) to develop a robust feature engineering pipeline that extracts predictive signals from diverse municipal data sources, (2) to create an interpretable machine learning model that provides building-level vacancy risk scores, and (3) to validate the predictive value of multi-source data integration compared to single-dataset approaches.

This work addresses a significant gap in urban analytics literature where few studies achieve meaningful integration of property, financial, transportation, regulatory, and economic datasets at building resolution. The research contributes to both academic understanding of urban economic dynamics and practical policy applications for city planners, real estate investors, and economic development agencies.

The study focuses specifically on NYC office buildings due to data availability, economic significance, and policy relevance. The methodology developed is transferable to other metropolitan areas with similar municipal data infrastructure.

\section{Literature Review}

\subsection{Real Estate Risk Modeling Foundations}

Vacancy prediction in commercial real estate builds upon established econometric foundations in hedonic pricing theory and urban economics. Rosen's seminal work on hedonic price models \cite{rosen1974hedonic} established the theoretical framework for understanding how building characteristics map to market values and occupancy rates. Subsequent research has extended these approaches to incorporate temporal dynamics and spatial dependencies in urban real estate markets.

Recent studies in real estate risk modeling have focused on time-series approaches for vacancy forecasting and absorption rate prediction. However, these models typically rely on aggregate market indicators rather than building-level characteristics, limiting their applicability for targeted policy interventions.

\subsection{Post-Pandemic Office Demand Analysis}

The COVID-19 pandemic has fundamentally altered office space demand patterns, creating new research imperatives for vacancy prediction. Barrero, Bloom, and Davis (2021) \cite{barrero2021working} provide comprehensive analysis of remote work adoption and its long-term implications for commercial real estate demand. Their findings suggest persistent structural changes in office utilization that traditional models fail to capture.

Studies of post-pandemic urban dynamics have highlighted the importance of transportation accessibility and neighborhood economic vitality as determinants of office desirability. However, limited research has operationalized these insights into predictive models using municipal administrative data.

\subsection{Municipal Data Integration in Urban Analytics}

Urban analytics research increasingly leverages administrative datasets to understand city dynamics. Studies utilizing NYC's PLUTO and ACRIS datasets have demonstrated the value of property and transaction data for neighborhood analysis \cite{furman2016fact}. However, integration across multiple municipal data sources remains challenging due to inconsistent identifiers, temporal misalignment, and scale differences.

Machine learning approaches to urban prediction problems have shown promise using tree-based models and ensemble methods. Random Forest and XGBoost algorithms have proven effective for tabular prediction tasks with heterogeneous feature sets, making them suitable candidates for multi-source municipal data integration.

\subsection{Spatial and Temporal Modeling Approaches}

Geospatial analysis in urban real estate has established the importance of proximity effects and neighborhood spillover patterns. Studies incorporating transportation accessibility typically use simple distance metrics, missing opportunities to leverage detailed ridership and usage patterns available in transit agency datasets.

Temporal modeling approaches in real estate focus primarily on time-series forecasting of aggregate metrics. Limited research has explored feature engineering from temporal patterns in individual building transaction histories or permit activity.

\subsection{Research Gaps and Contributions}

Current literature reveals several critical gaps: (1) limited integration of heterogeneous municipal datasets at building resolution, (2) insufficient incorporation of transportation demand patterns in vacancy prediction, (3) lack of explainable models that provide actionable insights for policy intervention, and (4) minimal validation of multi-source data integration benefits compared to traditional approaches.

This research addresses these gaps by demonstrating meaningful integration of six municipal datasets (PLUTO, ACRIS, MTA, DOB, Business Registry, Storefronts) with comprehensive feature engineering and explainable machine learning methods.

\section{Methodology}

\subsection{Data Integration Framework}

The methodology centers on integrating six heterogeneous NYC municipal datasets using Borough-Block-Lot (BBL) identifiers as the primary key for building-level analysis. The PLUTO (Primary Land Use Tax Lot Output) dataset serves as the canonical foundation, providing comprehensive building characteristics for 857,736 NYC properties.

\subsubsection{Dataset Integration Strategy}
\begin{itemize}
\item \textbf{PLUTO Integration}: Direct BBL-based joins for building characteristics, land use codes, and assessed values
\item \textbf{ACRIS Integration}: Temporal aggregation of transaction records by BBL with feature extraction from document types and transaction frequency
\item \textbf{MTA Integration}: Spatial proximity analysis linking buildings to nearest subway stations with ridership-weighted accessibility metrics
\item \textbf{DOB Integration}: Chunked processing of permit records with temporal aggregation by BBL
\item \textbf{Business Registry Integration}: Economic vitality indicators through business density and license status analysis
\item \textbf{Storefront Integration}: Neighborhood distress signals and ground truth vacancy indicators
\end{itemize}

\subsection{Feature Engineering Pipeline}

The feature engineering pipeline extracts meaningful predictive signals from each dataset through temporal aggregation, spatial analysis, and domain-specific transformations.

\subsubsection{Building Characteristics Features (PLUTO-derived)}
- Office area utilization ratios and size category classifications
- Building age categories and modernization requirements indicators  
- Floor Area Ratio (FAR) utilization and development potential metrics
- Zoning compatibility scores for office use optimization

\subsubsection{Financial Distress Indicators (ACRIS-derived)}
- Transaction velocity metrics (3-month, 6-month, 12-month rolling counts)
- Ownership change frequency and deed transfer pattern analysis
- Price trend slopes and transaction value volatility measures
- Mortgage activity patterns and refinancing frequency indicators

\subsubsection{Transportation Accessibility Features (MTA-derived)}
- Distance-weighted ridership accessibility scores
- Commuter flow pattern analysis and peak-hour accessibility
- Ridership trend analysis (2020-2024) with baseline comparison metrics
- Multi-modal transportation option density within 500m radius

\subsubsection{Investment Confidence Indicators (DOB-derived)}
- Construction permit frequency and estimated investment value
- Renovation activity patterns and building improvement trends
- Permit type analysis (major alterations vs. minor modifications)
- Temporal clustering of permit activity as investment confidence signals

\subsubsection{Economic Vitality Measures}
- Business license density and category diversity indices
- Commercial activity churn rates and new business formation patterns
- Neighborhood economic health indicators from business registry analysis
- Storefront vacancy clustering as neighborhood distress early warning signals

\subsection{Machine Learning Model Architecture}

The predictive model employs a gradient-boosted ensemble approach using XGBoost with SHAP (SHapley Additive exPlanations) for interpretability. The model architecture addresses class imbalance through stratified sampling and provides building-level risk scores with confidence intervals.

\subsubsection{Model Selection Rationale}
XGBoost was selected for its proven performance on heterogeneous tabular datasets, robust handling of missing values, and compatibility with SHAP explainability frameworks. The ensemble approach captures complex interactions between building, economic, and location factors while maintaining interpretability for stakeholder decision-making.

\subsubsection{Validation Strategy}
The validation approach employs temporal splitting to prevent data leakage, with training on historical data and validation on recent time periods. Geographic cross-validation ensures model generalization across NYC neighborhoods with different economic and demographic characteristics.

\subsubsection{Evaluation Metrics}
Model performance is assessed using ROC-AUC for overall discrimination, precision@k for actionable risk identification, and calibration metrics for probability interpretation. SHAP values provide feature importance rankings and individual prediction explanations for stakeholder transparency.

\subsection{System Architecture and Implementation}

The implementation framework supports scalable data processing, model training, and real-time risk scoring for operational deployment.

\subsubsection{Data Processing Pipeline}
- Automated ETL processes for municipal dataset updates
- Feature engineering automation with data quality monitoring
- Scalable processing architecture supporting NYC's 857K+ building universe

\subsubsection{Model Serving and Explainability}
- Batch scoring infrastructure for periodic risk assessment updates
- SHAP-based explanation generation for high-risk building identification
- Dashboard interfaces for stakeholder access to risk scores and explanations

The methodology demonstrates novel contribution through comprehensive municipal data integration, temporal feature engineering from administrative records, and explainable AI implementation for policy-relevant vacancy prediction.

\section{References}

\begin{thebibliography}{00}
\bibitem{rosen1974hedonic} S. Rosen, "Hedonic prices and implicit markets: product differentiation in pure competition," \textit{Journal of Political Economy}, vol. 82, no. 1, pp. 34-55, 1974.

\bibitem{barrero2021working} J. M. Barrero, N. Bloom, and S. J. Davis, "Why working from home will stick," \textit{National Bureau of Economic Research}, Working Paper 28731, 2021.

\bibitem{furman2016fact} J. Furman Center, "State of New York City's Housing and Neighborhoods," New York University Furman Center, 2016.

\bibitem{glaeser2018economic} E. L. Glaeser and J. Gyourko, "The economic implications of housing supply," \textit{Journal of Economic Perspectives}, vol. 32, no. 1, pp. 3-30, 2018.

\bibitem{been2019predicting} V. Been, I. Ellen, and J. Gedal, "Predicting the local impacts of HUD-assisted housing on property values," \textit{Cityscape}, vol. 21, no. 1, pp. 191-216, 2019.

\bibitem{kontokosta2017urban} C. E. Kontokosta and N. Johnson, "Urban phenology: Toward a real-time census of the city using Wi-Fi data," \textit{Computers, Environment and Urban Systems}, vol. 64, pp. 144-153, 2017.

\bibitem{athey2019machine} S. Athey and G. W. Imbens, "Machine learning methods economists should know about," \textit{Annual Review of Economics}, vol. 11, pp. 685-725, 2019.

\bibitem{cervero1997travel} R. Cervero and K. Kockelman, "Travel demand and the 3Ds: Density, diversity, and design," \textit{Transportation Research Part D}, vol. 2, no. 3, pp. 199-219, 1997.

\bibitem{zhang2013impact} M. Zhang and K. Wang, "The impact of mass transit on land value: A meta-analysis," \textit{Research in Transportation Economics}, vol. 40, no. 1, pp. 53-60, 2013.

\bibitem{elgeneidy2006access} A. El-Geneidy and D. Levinson, "Access to destinations: Development of accessibility measures," University of Minnesota, 2006.

\bibitem{ling1999fundamental} D. C. Ling and A. Naranjo, "The fundamental determinants of commercial real estate returns," \textit{Real Estate Economics}, vol. 27, no. 3, pp. 425-446, 1999.

\bibitem{case1989efficiency} K. E. Case and R. J. Shiller, "The efficiency of the market for single-family homes," \textit{American Economic Review}, vol. 79, no. 1, pp. 125-137, 1989.

\bibitem{yeh2018building} I. C. Yeh and T. K. Hsu, "Building real estate valuation models with comparative approach through case-based reasoning," \textit{Applied Soft Computing}, vol. 65, pp. 260-271, 2018.

\bibitem{liu2021machine} C. Liu and H. Wu, "Machine learning for real estate price prediction: A survey," \textit{IEEE Access}, vol. 9, pp. 123457-123478, 2021.

\bibitem{lundberg2020local} S. M. Lundberg, G. Erion, H. Chen, A. DeGrave, J. M. Prutkin, B. Nair, R. Katz, J. Himmelfarb, N. Bansal, and S. I. Lee, "From local explanations to global understanding with explainable AI for trees," \textit{Nature Machine Intelligence}, vol. 2, no. 1, pp. 56-67, 2020.

\bibitem{glaeser2017urban} E. L. Glaeser, J. Gyourko, and R. Saks, "Urban growth and housing supply," \textit{Journal of Economic Geography}, vol. 6, no. 1, pp. 71-89, 2017.

\bibitem{kahn2010evidence} M. E. Kahn, "New evidence on trends in the cost of urban agglomeration," in \textit{Agglomeration Economics}, University of Chicago Press, 2010, pp. 339-354.

\bibitem{bergmeir2012cross} C. Bergmeir and J. M. Benítez, "On the use of cross-validation for time series predictor evaluation," \textit{Information Sciences}, vol. 191, pp. 192-213, 2012.

\bibitem{lesage2009introduction} J. P. LeSage and R. K. Pace, \textit{Introduction to Spatial Econometrics}. Chapman and Hall/CRC, 2009.

\bibitem{chawla2002smote} N. V. Chawla, K. W. Bowyer, L. O. Hall, and W. P. Kegelmeyer, "SMOTE: Synthetic minority oversampling technique," \textit{Journal of Artificial Intelligence Research}, vol. 16, pp. 321-357, 2002.

\bibitem{kimball2013data} R. Kimball and M. Ross, \textit{The Data Warehouse Toolkit: The Definitive Guide to Dimensional Modeling}, 3rd ed. Wiley, 2013.

\bibitem{nyc_open_data} NYC Open Data, "PLUTO, ACRIS, MTA Ridership, DOB Permits, Business Registry, Storefronts," Available: https://opendata.cityofnewyork.us/, 2024.

\bibitem{chen2016xgboost} T. Chen and C. Guestrin, "XGBoost: A scalable tree boosting system," in \textit{Proceedings of the 22nd ACM SIGKDD International Conference on Knowledge Discovery and Data Mining}, 2016, pp. 785-794.

\bibitem{lundberg2017unified} S. M. Lundberg and S. I. Lee, "A unified approach to interpreting model predictions," in \textit{Advances in Neural Information Processing Systems}, 2017, pp. 4765-4774.
\end{thebibliography}

\end{document}